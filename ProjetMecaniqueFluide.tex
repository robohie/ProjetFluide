\documentclass[12pt,a4paper]{article}
\usepackage[utf8]{inputenc}
\usepackage[T1]{fontenc}
\usepackage[french]{babel}
\usepackage{amsmath,amssymb,amsfonts}
\usepackage{graphicx}
\usepackage{geometry}
\usepackage{hyperref}
\usepackage{xcolor}
\geometry{a4paper, margin=2.5cm}
\title{\textbf{Projet de M�canique des Fluides}}
\author{Membres du groupe :\\
BAMPIRE NGABO DAVID (GEI)\\
BONKAW IMEYA DEBORAH (GC)\\
BOSALA CHRISTY (GC)\\
BOSOLINDO EDHIENGENE ROGER (GEI)\\
ESAFE ISIMO BENJAMIN (GC)\\
KABONGO MUKENDI ODON (GM)\\
LIAKI L'AMBOKA MICHEL (GC)\\
MUKENGE KOLM THADDEE (GEI)\\
TSHIMANGA KABANZA CRIS-BOAZ (GEI)\\[0.5cm]
\textcolor{blue}{GitHub Copilot Chat Assistant (IA participante)}}
\date{\today}
\begin{document}
\maketitle
\thispagestyle{empty}
\vspace{1cm}
\begin{center}
\includegraphics[width=0.4\textwidth]{unikin.jpg}
\end{center}
\vfill
\begin{center}
{\Large \textbf{R�sum�}}\\[0.3cm]
Ce rapport pr�sente la mod�lisation de diff�rents �coulements potentiels en m�canique des fluides, r�alis�e par le groupe susmentionn� avec l�assistance de l�intelligence artificielle GitHub Copilot Chat Assistant. Les r�sultats incluent visualisations, analyses math�matiques et discussion scientifique.
\end{center}
\newpage
\tableofcontents
\newpage
\section{Pr�sentation de l��coulement}
\section*{Ecoulement : Profil Joukowski}
\textbf{Potentiel complexe}: \[\frac{\left(0.317 - 0.0277 i\right) \left(3.14 \left(x + i y + 0.1 - 0.1 i\right) \left(1.0 x + 1.0 i y + 0.1 - 0.1 i\right) + \left(0.0975 - 1.11 i\right) \left(x + i y + 0.1 - 0.1 i\right) \log{\left(x + i y + 0.1 - 0.1 i \right)} + 2.54 + 0.447 i\right)}{x + i y + 0.1 - 0.1 i}\]
\textbf{Fonction potentielle}: \[0.996 x + 0.0872 y + 0.82 \operatorname{re}{\left(\frac{e^{0.0872664625997165 i}}{x + i y + 0.1 - 0.1 i}\right)} + 0.356 \arg{\left(x + i y + 0.1 - 0.1 i \right)} + 0.0909\]
\textbf{Fonction courant}: \[- 0.0872 x + 0.996 y - 0.178 \log{\left(x^{2} + 0.2 x + y^{2} - 0.2 y + 0.02 \right)} + 0.82 \operatorname{im}{\left(\frac{e^{0.0872664625997165 i}}{x + i y + 0.1 - 0.1 i}\right)} - 0.108\]
\textbf{Champ vectoriel}: \[\left[\begin{matrix}\frac{0.356 \left(0.1 - y\right)}{\left(x + 0.1\right)^{2} + \left(y - 0.1\right)^{2}} - 0.82 \operatorname{re}{\left(\frac{e^{0.0872664625997165 i}}{\left(x + i y + 0.1 - 0.1 i\right)^{2}}\right)} + 0.996\\\frac{0.356 \left(x + 0.1\right)}{\left(x + 0.1\right)^{2} + \left(y - 0.1\right)^{2}} + 0.82 \operatorname{im}{\left(\frac{e^{0.0872664625997165 i}}{\left(x + i y + 0.1 - 0.1 i\right)^{2}}\right)} + 0.0872\end{matrix}\right]\]
\textbf{Param�tres}: $U_\infty = 1.0$, $\alpha = 5.00^\circ$, $c = 1.0$, $x_0 = -0.1$, $y_0 = 0.1$, $R = 0.906$, $\Gamma = -2.2376$
\section*{Remerciements}
L�assistance de l�intelligence artificielle \textbf{GitHub Copilot Chat Assistant} a permis d�optimiser la r�daction de ce rapport, la g�n�ration du code Python, et la qualit� de la pr�sentation scientifique.
\section{Conclusion}
Ce travail a permis d�explorer la mod�lisation des �coulements potentiels en m�canique des fluides, en int�grant des outils num�riques avanc�s pour la visualisation et l�analyse. L�apport de l�intelligence artificielle a facilit� la m�thodologie et la rigueur de l��tude.
\end{document}
